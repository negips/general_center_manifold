\section{Introduction}

Center-manifold theory provides an elegant and powerful tool for the analysis of dynamics of systems in the vicinity of the bifurcation point \citep{carr82,carr83b,sijbrand85,guckenheimer83,wiggins03}. In such systems the essential dynamics are largely driven by the dynamics in the center-manifold, which is a continuation of the center space of the linearized system into the non-linear regime. Being a continuation of the center space, the center manifold has the same dimension as the center space and leads to a low-dimensional representation of what might originally be a very high (or infinite) dimensional system. The ideas underlying the theory are powerful and extend beyond center manifolds, having found applicability to general invariant manifolds \citep{roberts89}. In the context of hydrodynamics, the investigations of such systems is often done by invoking the idea of separation of rapid oscillations from slowly modulating amplitudes \citep{stuart60,watson60,newell69}. In the more recent literature it is often referred to as weakly non-linear analysis and follows the formal approach of multiple-scale analysis \citep{sipp07,meliga11,meliga12}. The procedure results in a set of equations for the slowly varying part of the amplitudes of the unstable eigenmode(s), referred to as the amplitude equations \citep{cross09} or sometimes the Stuart-Landau equations. While it has been claimed that the two methods are effectively equivalent \citep{fujimara91}, the underlying ideas of the two methods are somewhat distinct, leading to contrasting procedures for the formal evaluation of the approximations. As opposed to the method of multiple-scales, the center manifold theory assumes no separation of time scales and divides the solution space into a ``dynamic'' component corresponding to center space of the linearized system at bifurcation and a ``driven'' component corresponding to the stable subspace. The center-manifold theorem leads to a graph equation for the driven component of the solution. While the application of center-manifold theory to ordinary differential equation type problems is fairly straightforward the application to partial differential equations, which are infinite dimensional, poses a harder challenge since the graph equation is a highly non-linear, infinite-dimensional problem. 

In the literature the approach to the problem appears to converge to computing the normal form of the reduced center-manifold system \citep{knobloch83,guckenheimer83b,coullet83,haragus11}. Differences arise in the flavor of the approach in the various studies. For example simple self-adjoint problems were investigated in \cite{carr83} where the solution to the graph equation was shown to vanish at second order, resulting in simplified evaluation of the center manifold. Sometimes reasonable justification may be provided for truncating the stable subspace to finite dimension \citep{knobloch83,guckenheimer83b,fujimura97}, in which case the center-manifold approximation can effectively be treated in manner similar to ordinary differential equations. An iterative numerical approach has been pursued in \cite{roberts97} which sought to minimize the residual of the amplitude equation for the critical modes to determine the total solution. A similar algorithmic approach was undertaken in \cite{carini15} for the approximation of normal forms for the center manifolds. The variations not withstanding, the main idea commonly utilized is to approximate the non-linear transforms required for achieving the normal form for the low-dimensional (center-manifold) system, with Taylor expansions used to treat variation of the system parameters \citep{haragus11}. 

The current work takes a different approach for the evaluation of the center-manifold, falling back to the spirit of its computation for ordinary differential equations \citep{wiggins03}, while also accounting for the infinite-dimensional nature of the problem. This has been done in the context of the Navier--Stokes however, the underlying idea may easily be adapted to other systems. Taking a step back, the Navier--Stokes system is first extended to include the trivial equation for the parameter evolution, with the parameter now treated as a dynamic variable. This increases the dimension of the center-manifold which has non-trivial consequences as shall be seen. This requires understanding how the spectral properties of the extended system relate to those of the original problem, which is done in a general setting. The system is then transformed into an appropriate form and the center-manifold theorem is then applied to this modifed problem leading to an infinite-dimensional graph equation, which is solved asymptotically. The graph equation and the subsequent asymptotic solutions are formally valid for systems that have been perturbed away from the bifurcation point, as opposed to the case of the center-manifold reduction of the standard problem where the graph equation is valid only at the bifurcation point. A major consequence of this approach is that when the system is perturbed away from the bifurcation point one does not need to consider the Taylor expansions with respect to the system parameter(s). The solution is essentially baked in to the asymptotic expressions that are derived. The primary focus of the work then is not so much on the derivation of the normal forms but rather in the evaluation of the graph equations. Once the solutions to the graph equations are known, the subsequent step of obtaining the ``amplitude equations'' is obtained almost as a byproduct. In this sense, the current work may also be viewed as an alternate route for a rigorous derivation of the Landau model \citep{landau_52} in infinite-dimensional systems, which is often used to understand the dynamics of weakly non-linear systems. 

The remainder of the manuscript is laid out as follows. Section~\ref{problem_setup} first describes the notation used in the current manuscript, then presents the problem in a general setting of the Navier--Stokes equation and subsequently explores the consequences of extending the system with the trivial parameter evolution equation. Section~\ref{center_manifold_derivation} then reformulates the problem in an appropriate form for the application of center-manifold theory. An asymptotic approximation is then carried out and the resulting expressions for the power series are obtained. The presented theory is then applied to the case of a Hopf bifurcation in a cylinder wake in section~\ref{application_cylinder} and to the case of the flow in an open cavity in section~\ref{application_cavity}. Concluding remarks are made in section~\ref{conclusion}.










