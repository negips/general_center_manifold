% for Swedish letters remove the option 'english'
\documentclass[english]{kthletter}

% Define here relevant information
% defines eventual bank accounts (usually left blank)
\accounts{}
% your address
\address{Osquars Backe 18, 100 44 Stockholm}
% the date
\date{Stockholm, \today}
%\location{Lindstedtsvägen 24}
\name{P. S. Negi}
\signature{P. S. Negi, M. Mishra, P. Schlatter and M. Skote}
\telephone{+46 8 790 7894}
\telefax{+46 8 790 7854}
\email{negi@mech.kth.se}
%\web{https://www.kth.se/profile/giampi/}

\begin{document}

\begin{letter}[]% optional dossier number
  {
  	Physical Review Fluids,\\
  	Editorial Office\\
%  	Profs. Pollard and Jakirlic
  	}

\opening{Dear Editor,}

Please find enclosed the paper titled “Bypass transition delay using oscillations of spanwise wall velocity” which we would like to submit as a journal paper to the Physical Review Fluids.

The paper reports the results of a systematic study of the application of oscillatory spanwise wall velocity on the phenomenon of bypass transition of boundary layers. The results are reported for both temporal as well as spatial oscillations of the spanwise wall-velocity with the primary focus being on the control using spatial oscillations. Variation of transition delay is studied individually for several control parameters such as oscillation amplitude, wavenumber (frequency), freestream turbulence intensity and the spatial extent of control. Also reported are the effects of control on fluctuation quantities.

These results are put in the context of the phenomenon of shear-filtering of the continuous Orr-Sommerfeld modes and it is shown that the Stokes' layer formed due to the spanwise oscillations creates a secondary filtering effect inside the boundary layer which selectively affects the penetrating OS modes. The effect of the control is further demonstrated using the 2-mode model of bypass transition.

We believe that the insights gained from the analysis are of great interest both for practitioners and academics. We hope that you will find this manuscript suitable for publication in the International Physical Review Fluids.

\closing{Best regards,}



\end{letter}
\end{document}
